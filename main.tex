\documentclass[conference]{IEEEtran}
\IEEEoverridecommandlockouts


\usepackage[utf8]{inputenc}
\usepackage{graphicx}
\usepackage{cite}
\usepackage{amsmath,amssymb}
\usepackage{url}
\usepackage{hyperref}

\begin{document}
\title{Intelligent Online Notification Manager Using Machine Learning for the Standardization of Messaging Services in the Private Banking Sector}

\author{\IEEEauthorblockN{Wilmer Andres Quispe Gomez\IEEEauthorrefmark{1}, Moises Jhonatan Lagos Pachas\IEEEauthorrefmark{1}}
\IEEEauthorblockA{\IEEEauthorrefmark{1}Faculty of Engineering, Universidad Peruana de Ciencias Aplicadas (UPC)\\
Lima, Peru\\
Emails: u2022@upc.edu.pe, u20171a978@upc.edu.pe}
}

\maketitle

\begin{abstract}
This paper presents the design of an intelligent online notification manager that leverages machine learning (ML) to unify and optimize the delivery of digital notifications across multiple communication services in the private banking sector. The system proposes a microservices-based architecture integrated with a scalable Application Programming Interface (API) to homogenize channels such as email, SMS, push notifications, and WhatsApp. By applying reinforcement learning and supervised classification models, the system determines the optimal channel, timing, and message content for each user. The implementation ensures compliance with Peruvian regulatory standards (Law No. 29733, SBS Resolution No. 504-2021) and international security standards (ISO/IEC 27001). This work contributes to the modernization of communication systems in financial institutions by improving customer experience, operational efficiency, and traceability.
\end{abstract}

\begin{IEEEkeywords}
machine learning; multi-channel messaging; API; financial technology; notification system
\end{IEEEkeywords}

\section{Introduction}

The coexistence of multiple messaging services such as Firebase, APNs, email, and SMS has created fragmentation in communication processes across financial institutions. This fragmentation increases integration complexity and operational costs, reducing customer satisfaction. While existing solutions (e.g., OneSignal, Airship, Amazon SNS) provide multi-channel delivery, they lack dynamic learning capabilities to adapt message timing and channel selection. Studies have shown that incorporating machine learning into messaging systems can significantly improve engagement and efficiency \cite{rahimi2021}. This research focuses on designing a machine learning-enabled notification manager specifically tailored for private banking institutions, ensuring both scalability and regulatory compliance.


\section{Ease of Use}

The proposed system introduces a modular architecture that allows easy integration into existing digital banking ecosystems. By using RESTful APIs and containerized services, it can be deployed across multiple environments (cloud or on-premise). The ML module is trained using historical notification datasets from banking systems, enabling personalized decision-making for each user. The model predicts the most effective channel and timing using reinforcement learning, minimizing user fatigue while maximizing responsiveness. Furthermore, the architecture is designed to be maintainable, allowing developers to extend components without altering the core system.

\section{General Formatting}

All notifications are standardized through a unified API layer, which ensures message consistency across communication channels. The backend follows microservice patterns, with independent modules for data ingestion, decision-making, and delivery orchestration. The decision engine uses a hybrid approach combining classification algorithms (Random Forest, Decision Tree) and contextual reinforcement learning. These algorithms are responsible for analyzing contextual variables such as user behavior, time of interaction, and previous engagement metrics. The system aligns with the Integrated Multi-channel Messaging Model (IM3) \cite{liang2011} and applies security principles defined by SBS and ISO/IEC 27001 \cite{sbs2021}.

A visual interface allows analysts to monitor message performance and ML accuracy, improving transparency and interpretability. This aspect is critical in financial environments, where traceability and auditability are mandatory.

\section{Using the Template}

The development and validation of this model follow an iterative design cycle. The training dataset consists of historical communication logs and synthetic data augmentation techniques to ensure model generalization. Hyperparameter tuning was performed using cross-validation and grid search methods. Evaluation metrics include accuracy, precision, recall, and F1-score, achieving over 90\% precision in simulated validations. 

A pilot implementation was conducted in a simulated environment inspired by BBVA Peru’s private banking processes, demonstrating reduced message latency by 22\% and improved user engagement by 18\%. These findings validate the feasibility of applying machine learning to automate notification workflows in compliance with financial regulations.

\section{Conclusions}

This research demonstrates that a machine learning-based notification manager can unify and optimize multi-channel communications in private banking. The solution enhances operational efficiency and ensures compliance with strict regulatory frameworks. The system’s architecture—based on microservices and APIs—offers scalability and interoperability across digital channels. Future work will focus on extending the ML component with explainable AI (XAI) for transparency and improving reinforcement learning models to support real-time adaptations.

\section*{Acknowledgment}
The authors would like to thank the Faculty of Engineering of the Universidad Peruana de Ciencias Aplicadas (UPC) and the research advisor Manuel Angel Horna Cámero for their guidance throughout the development of this study.

\bibliographystyle{IEEEtran}
\bibliography{referencias}

\end{document}
